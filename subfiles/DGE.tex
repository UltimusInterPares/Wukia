\chapter{Selections from Schwyzer, Eduard: \textit{Dialectorum Graecarum Exempla Epigraphica
Potiora}}

\section{Tituli Megarici}

\subsubsection{DGE 148}
\textit{Prope Megara.  \textgreek{Πλὰξ} marmorea.  \textgreek{στοιχ.}  V in. — Tit. sepulcralis. — Wilhelm AM 31, 89/93. Im 52. So 26. Kern t. 9. Ge 89.}
\begin{versi}{DGE 148}
\begin{greek}
\ladd{Λα}{\K}κλ\~{ε} τὸν Προκ{\verso[1]}λέος· ται δ’ ἐνπίδες, αἴ τέ κα ἀ(λ)λ\~{ε} \; καἴ κ’ {\verso} ἀ(λ)λ\~{ε}, θαψ\~{ε}ν τε̃δε τρ{\verso}όπ\={ο}ι πό\ladd{λιο}{\K}ς.
\end{greek}
\end{versi}

\hrulefill

\subsubsection{DGE 149}
\textit{Dedicationes. — 1) [²100] In Megaridis vico \textgreek{Μαζί}. Titulus grandibus litteris incisus in lapide rudi. — Lebègue, De oppidis Megaridis p. 37. Im 52. Kirchh. A. 112. Rob 213. IG VII 35. Mi 755. — 2) Prope Megara. Tabella aenea. — Korolkow AM 8, 181/3. Kirchh. A. 113. DI 3001. IG VII 37. Im 53. Rob 113a. MI 1062. Ge 71.}
\begin{versi}{DGE 149.1}
\begin{greek}
Ἀπό(λ)λ\={ο}νος {\verso[1]} Λυκεί\={ο} —
\end{greek}
\end{versi}

\begin{versi}{DGE 149.2}
\begin{greek}
τ{\rbrk}οίδε ἀπὸ λα\ladd{ία}{\verso[1]}ς τὰν δεχάτα\ladd{ν} {\verso} ἀνέθ\={ε}καν Ἀθά{\verso}ναι
\end{greek}
\end{versi}
\textit{\textgreek{ε} = B. \textgreek{ε, \={ε}} = B; metr.}

\hrulefill

\subsubsection{DGE 150}
\textit{Piraei. Stela marmorea. — Tit. sepulcralis. Lenormant RhM 21, 376.  \textgreek{Κουμανούδης Ἀ. ἐ. ἐ. } 2008. IGA 13. Kirchh. A. 113. Rob 114. DI 3002.}
\begin{versi}{DGE 150}
\begin{greek}
Ἀπολλόδ{\verso[1]}\={ο}ρος \; Διο{\verso}κλΕδα {\verso} ΜΕγαρΕύς.
\end{greek}
\end{versi}

\hrulefill

\subsubsection{DGE 151}
\textit{Prope Megara. Fragmenta vasis fictilis. — Dedicatio. — \textgreek{Φίλιος} EA 1890, 45. IG VII 3493. Im 54.}
\begin{versi}{DGE 151}
\begin{greek}
Ε{\rbrk}ὐκλΕδας καὶ ΜhΕιλο – –
\end{greek}
\end{versi}

\hrulefill

\subsubsection{DGE 153}
\textit{Megaris prope Olympieum. Lapis calcarius. Paullo post 301. — Proxenia. — Heath ABS 19, 82/8 (III)}
\begin{versi}{DGE 153}
\begin{greek}
ἐπὶ βασιλέος Ἀντιφίλου, {\verso[1]} γραμματεὺς βουλᾶς, καὶ δάμ\ladd{ου} {\verso} Ἀπολλόδωρος Εὐπαλίνου, ἐστ\ladd{ρα}{\verso}τάγουν Φωκῖνος Εὐάλχου, Ἀρισ\ladd{τό}{\K}τιμος Μενεκράτεος. Δαμοτέλ\ladd{ης} {\verso} Δαμέα, Θέδωρος Παγχάρεος, Πρ\ladd{ό}{\verso}θυμος Ζεύξιος, Τίμων Ἀγάθων\ladd{ος}{\verso}, ἔδοξε βουλᾶι καὶ δάμωι· {\verso} ἐπειδὴ Λυκίσκος Φυσάου Ἀλε\ladd{ῖος} διατελεῖ εὔνους ἐὼν τῶι δάμ\ladd{ωι} {\verso} τῶι Μεγαρέων, δεδόχθαι τᾶι βου\ladd{λᾶι} {\verso} καὶ τῶι δάμωι πρόξενον αὐτὸν \ladd{εἶ}{\verso}μεγ καὶ εὐεργέταν τᾶς πόλιος \ladd{τᾶς} {\verso} Μεγαρέωγ καὶ εἶμεν αὐτῳι γᾶς καὶ οἰκίας ἔμπασιγ καὶ προεδρί\ladd{αν} {\verso} ἐμ πᾶσι τοῖς ἀγῶσιν οἷς ἁ πόλις \ladd{τίθη}{\verso}τι· ἀγγραψάτω δὲ τὸ δόγμα τόδε ὁ \ladd{γραμ}{\verso}μαεὺς τοῦ δάμου ἐν στάλαι λιθ\ladd{ίναι} {\verso} καὶ ἀνθέτω εἰς τὸ Ὀλυμπιεῖον. 
\end{greek}
\end{versi}

\hrulefill

\subsubsection{DGE 154}
\textit{Megaris prope Olympieum. Lapis leucophaneus cum aëtomate. Paullo post 306. — Honores. — Korolkow AM 8, 183/9. DI 3010. IG VII 1. Mi 166. DS 174 ³331.}
\begin{versi}{DGE 154}
\begin{greek}
ἐπὶ βασιλέος Ἀπολλοδώρου τοῦ Εὐφρονίου, γραμματεῦς βουλᾶς {\verso[1]} καὶ δάμου Δαμέας Δαμοτέλεος, ἐστρατάγουν Δαμοτέλης {\verso} Δαμέα, Φωκῖνος Εὐάλκου, Ἀριστότιμος Μενεχράτεος, Θέδωρος {\verso} Παγχάρεος, Πρόυθμος Ζεύξιος, Τίμων Ἀγάθωνος· ἐπειδὴ τοὶ Αἰγοστενῖτα\ladd{ι} ἀνάγγελλον Ζωΐλογ Κελαίνου Βοιώτιον {\verso} τὸν ἐπὶ τοῖς στρατιώταις τοῖς ἐν Αἰγοστένοις τεταγμένον ὑπὸ τοῦ {\verso} βασιλέος Δαματρίου, αὐτόν τε εὔτακτον εἶμεν καὶ τοὺς στρατιώ{\verso}τας παρέχειν εὐτάκτους καὶ τἆλλα ἐπιμελεῖσθαι καλῶς καὶ εὐ{\verso}νόως καὶ ἀξίουν διὰ ταῦτα τιμαθῆμεν ὑπὸ τᾶς πόλιος, ἀγαθᾶι τύχαι, δεδόχθαι τᾶι βουλᾶι καὶ τῶι δάμωι, στεφανῶσαι {\verso} Ζωΐλογ Κελαίνου Βοιώτογ χρυσέωι στεφάνωι καὶ εἶμεν αὐτὸμ πο{\verso}λίταν τᾶς πόλιος τᾶς Μεγαρέωγ καὶ ἐκγόνους αὐτοῦ· εἶμεν δὲ {\verso} αὐτῶι καὶ προεδρίαν ἐμ πᾶσι τοῖς ἀγῶσι οἷς ἁ πόλις τίθητι. ἀγγράψαι δὲ τόδε τὸ δόγμα τὸγ γραμματέα τοῦ δάμου εἰς στάλαν λιθίναν {\verso} καὶ ἀνθέμεν εἰς τὸ Ὀλυμπιεῖον, ὅπως εἰδῶντι πάντες, ὅτι ὁ δᾶμος {\verso} \ladd{ὁ Μ}{\K}εγαρέων τιμῆ τοὺς ἀγαθόν τι πράσσοντας ἢ λόγωι ἢ ἔργωι {\verso} ὑπὲρ τᾶς πόλιος ἢ ὑπὲρ τᾶγ χωμᾶν.
\end{greek}
\end{versi}
\textit{vss. 5. 6 \textgreek{Αἰγόστενα} (caprarum angustiae) est antiquior forma quam usitata \textgreek{Αἰγόσθενα} (IG).}

\hrulefill

\subsubsection{DGE 155}
\textit{[²106] Megaris in muro Olympiei. Lapis. III in. — Proxenia. — Ross BerlBer 1844, 160 sq. DI 3005. IG VII 8.}
\begin{versi}{DGE 155}
\begin{greek}
ἐπιδὴ Ἀγαθοκλῆς Ἀρχιδάμου {\verso[1]} Βοιώτιος εὔνους ἐὼν διατελεῖ {\verso} καὶ εὐεργέτας τοῦ δάμου τοῦ {\verso} Μεγαρέων, ἀγαθᾶι τύχαι δεδόχθαι τᾶι βουλᾶι καὶ τῶι δάμωι· {\verso} πρόξενον αὐτὸν εἶμεγ καὶ ἐκ{\verso}γόνους αὐτοῦ τᾶς πόλιος τᾶς {\verso} Μεγαρέωγ κὰτ τὸν νόμον· εἶμεν {\verso} δὲ αὐτῶι καὶ οἰκίας ἔμπασιν καὶ προεδρίαν ἐμ πᾶσι τοῖς ἀγῶ{\verso}σιν οἷς ἁ πόλις τίθητι· ἀγγραψά{\verso}τω δ\~{ε} τ\~{ο} δόγμα τόδε ὁ γραμματεὺς τοῦ δάμου ἐν στάλαι λιθί{\verso}ναι καὶ ἀνθέτω εἰς τὸ Ὀλυμπιεῖον. βασιλεὺς Πασιάδας· ἐστρατά{\verso}γουν Διονύσιος Πυρρίδα, Δαμέ{\verso}ας Ματροκλέος, ἀντίφλιλος Σμά{\verso}χου, Μνασίθεος Πασίωνος, Ἑρκίων {\verso} Τέλητος· γραμματεὺς βουλᾶς καὶ δάμου Ἵππων Παγχάρεος.
\end{greek}
\end{versi}

\hrulefill

\subsubsection{DGE 158}
\textit{[²108] Megaris in domo privata. Post a. 192, aliquot annis ante a. 159. — Honores. — Velsen (et E. Curtius) AZ 11, 379/83. Le Bas II 35a. Vischer KlSchr 2, 64 9. DI 3016. IG VII 15. Mi 169. DS 297. Cf. Wilhelm GöttAnz 1900, 104.}
\begin{versi}{DGE 158}
\begin{greek}
συναρχίαι προεβουλεύσαντο ποτί τε τοὺς αἰσιμνάτα\ladd{ς {\verso[1]} καὶ τὰν} βουλὰν καὶ τὸν δᾶμον· ἐπειδὲ Ἱκέσιος Μητροδ\ladd{δώ{\verso}ρου} Ἐφέσιος ὁ κατασταθεὶς ἐπ’ Αἰγίνας ὑπ\ladd{ὸ τοῦ {\verso} βασ}{\K}ιλέος Εὐμένεος διατελεῖ τὰν πᾶσαν σ\ladd{που}{\K}δὰν \ladd{ποιούμενος ὑπὲρ} τοῦ δάμου το\ladd{ῦ} Μ\ladd{ε}{\K}γαρέων, \ladd{ο}{\K}ὐθὲν \ladd{ἐλλείπων} - - τῶι δ\ladd{άμωι} - -.
\end{greek}
\end{versi}

\hrulefill

\subsubsection{DGE 159}
\textit{Megaris. Lapis. 223/192. — Conscriptio epheborum. — Le Bas II 34a. DI 3020. IG VII 27. Mi 618.}
\begin{versi}{DGE 159}
\begin{greek}
ἄρχοντος Κλειμάχου, {\verso[1]} ἐν δὲ Ὀγχηστῶι Ποτιδαΐχου, {\verso} ἐπολεμάρχουν {\verso} Ἡράκλειτος Κνιφᾶ, Ἀπολλόδωρος Φίλωνος, {\verso} Ἀριστόνικος Ἀγαθοκλέος, {\verso} Ἀλκαῖος Τελεσία, {\verso} Δαμάτριος Εὐδάμου, {\verso} τοίδε ἀπῆλθον ἐξ ἐφήβων εἰς τὰ τάγματα· Ἀγέλαος Ἡροτίμου, {\verso} Ἀρίστων Μυρτίλου, {\verso} Μέγυλλος Κορίβου, {\verso} Πολυδευκείδας Θεδώρου, Ἀγόλαος Ἀριγνώτου, {\verso} Ἀσκλαπίων Νίκωνος, {\verso} Χαρίδαμος Διοδώ\ladd{ρο}{\K}υ, {\verso} Ἡ\ladd{ρ}{\K}όδοτος Θεδώρου, {\verso} Ματρόδωρος Ἀντιφίλους, Ἀγέμαχος Χαριλάου, {\verso} ……ος Παιανίχου, {\verso} Μελ\ladd{ισσ}{\K}ίων Φιλλέα, {\verso} Ἀπολλόδωρος Ὀλυμπίχους, {\verso} Διοκλείδας Χαριδάμου, Μελισσίων Ματροξένου, {\verso} Σωκλ\ladd{ί}{\K}ας Ἡράχων\ladd{ος}.
\end{greek}
\end{versi}
\textit{``Aetate qua Megarenses cum Boeotiis foedere coniuncti erant'' (IG).}

\hrulefill

\subsubsection{DGE 160}
\textit{Megaris. 30/27. — Honores. — CIG 1069 (e Sponii Misc.). DI 3019. IG VII 63.}
\begin{versi}{DGE 160}
\begin{greek}
ὁ δᾶμος {\verso[1]} αὐτοκράτορα Καί{\verso}σαρα θεοῦ υἱὸν {\verso} ἀρετᾶς ἕνεκεν {\verso} καὶ εὐνοίας
\end{greek}
\end{versi}

\hrulefill

\subsubsection{DGE 161}
\textit{Aegosthenis. Lapis tria continens decreta in porta ecclesiae \textgreek{τῆς Παναγίας} ita inaedificatus, ut limen superius efficiat. 223/201. — Honores. — Forchhammer Halkyonia 1857, 30/3. Le Bas II 2. DI 3091. IG VII 208. Mi 171.}
\begin{versi}{DGE 161}
\begin{greek}
Νικίας Διονυσίου ἔλ\ladd{ε}{\verso[1]}ξε· προβεβωλευμένον εἶμεν αὐτοῖ· ἐπειδὲ Πολέμα\ladd{ρ}{\verso}χος Μένωνος Χαλεῦς εὐεργέτας ἐὼν διατελῖ {\verso} τᾶς πόλιος Αἰγοσθενιτᾶν, ὑπάρχειν δὲ α᾽τῶι καθάπερ {\verso} καὶ τοῖς ἄλλοις προξένοις καὶ εὐεργέταις. τ\ladd{οὶ} δὲ πολέμαρχοι, ἐπεί κα τὸ ψάφισμα χυρωθε\ladd{ίη}, {\verso} ἀνγραψάντω εἰς στάλαν ἐν τοῖ Μελαμποδείοι.
\end{greek}
\end{versi}
\textit{Aetatis qua Aegosthenitae Boeotorum foederis erant socii (223/192). Praeter initium Boeoticae dialecti vestigia sunt \textgreek{διατελῖ} (vs. 3) et dativi in \textgreek{-οι} cadentes; cf. praeterea \textgreek{ἐν} c. acc. (nr. 162; cf. Sadée 28, 2), \textgreek{ἀπε\ladd{γρ}{\K}άψανθο} IG VII 214, 2. Totum Boeotica dialecto compositum est decretum Aegosthenitarum nr. 450. vs. 8 \textgreek{χυρωθε\ladd{ίη}} DI; probavit Solmsen RhM 63, 334.}

\hrulefill

\subsubsection{DGE 162}
\textit{Aegosthenis. In eodem lapida ac nr. 161. 220/200. — Conscriptio epheborum. — Le Bas II 4. DI 3096. IG VII 210. Mi 621.}
\begin{versi}{DGE 162}
\begin{greek}
Ὀνασίμου ἄρχοντος ἐν Ὀγχηστῶι, ἐξ ἐ\ladd{φή}{\verso[1]}βων ἐν πελτοφόρας ἀπεγράψατο Ἀλχίας {\verso} Ἀπολλοδώρου.
\end{greek}
\end{versi}

\hrulefill

\subsubsection{DGE 163}
\textit{[²104] Lapis olim in ecclesia \textgreek{τοῦ ἁγίου} Γεωργίου prope Aegosthena, 192/146. — Proxenia. — Leake, North. Gr. 2, 405. Le Bas II 12. DI 3094. IG VII 223. Mi 172.}
\begin{versi}{DGE 163}
\begin{greek}
ἀγαθᾶι τύχαι. ἐπὶ γραμμ\ladd{ατ}{\K}έω\ladd{ς} ..., {\verso[1]} ἐπὶ δὲ βασιλέως ἐν Αἰγοσθ\ladd{ένοις Ἡρά}{\verso}χωνος, μηνὸς τρίτου, συναρχί\ladd{αι προ}{\verso}εβουλεύσαντο ποτὶ τὰν βουλ\ladd{ὰν} καὶ τὸν δᾶμον· ἐπειδῆ Ἀπολ\ladd{λόδω}{\verso}ρος Ἀλκιμάχου Μεγαρεὺς εὔ\ladd{νους} {\verso} ἐὼν διατελεῖ τῶι δάμωι τ\ladd{ῶι Αἰ}{\verso}γοσθενιτᾶν καὶ χρείας παρέ\ladd{χεται} {\verso} καὶ κοινᾶι καῖ καθ’ ἱδίαν τοῖς δε\ladd{ομέ}{\K}νοις τῶν πολιτᾶν, ἀεί τινος ἀ\ladd{γαθοῦ {\verso} π}{\K}α\ladd{ραίτι}{\K}ος γενόμενος· ἀγαθᾶι τύχ\ladd{αι δεδό}{\K}χθαι τᾶι \ladd{βουλᾶι κα}{\K}ὶ τῶι δάμωι, πρό\ladd{ξε}{\verso}νον εἶμεν αὐτὸν καὶ \ladd{ἐκ}{\K}γόν\ladd{ους} {\verso} τᾶς πόλιος Αἰγοσθενιτᾶν· εἶ\ladd{μεν δὲ} αὐτῶι ἔγκησιν γᾶς καὶ οἰκία\ladd{ς καὶ} {\verso} τὰ ἄλλα πάντα, ὅσα καὶ τοῖς ἄλλ\ladd{οις} {\verso} π\ladd{ρ}{\K}οξένοις ὁ \ladd{νό}{\K}μο\ladd{ς} κε\ladd{λ}{\K}εύει· εἶμε\ladd{ν δὲ} {\verso} αὐτῶι καὶ ἐπινομίαν· ἐπεὶ δέ κα \ladd{δό}{\verso}(ξ)η, ἀναγραψάντω οἱ δ\ladd{αμ}{\K}ιοργοὶ εἰ\ladd{ς στά}{\K}λαν λιθίναν ἐν τῶι ἱ\ladd{ερῶι} τοῦ Με\ladd{λάμ}{\verso}ποδος· δίδοσθαι δὲ \ladd{καὶ} μερίδα α\ladd{ὐτῶι ἐ}{\verso}κ τῶν Μελαμποδείων, καὶ καλ\ladd{έσαι} {\verso} αὐτὸν εἰς προεδρίαν καθάπερ \ladd{κα{\verso}ὶ το}{\K}ὺς ἄλλους προξένους.
\end{greek}
\end{versi}

\section{Acta foederis Boeotici et tituli incertae originis}

\subsubsection{DGE 440}
\textit{Titt. vasorum fictilium et aenei (11). VI — Perdrizet BH 20, 242sq. (11). \textgreek{Σταυρόπουλλος} EA 1896, 243/6 (1. 2. 3. 11).  \textgreek{Κουρουνιώτης} ib. 1900, 101/10 (6. 10. 12). *Harvard Studies 2, 89/101 (4). Kr.V. 52/4 (7. 8). Pollak RM 2, 105/11 (9). IG VII 1685 (5). 1873 (7). Wolters AM 38, 193sq. (13; cf. Kretschmer Gl 7, 333).}
\begin{versi}{DGE 440.2}
(Tanagrae)
\begin{greek}
Μαϙυταέα ε̄̓μί.
\end{greek}
\end{versi}

\begin{versi}{DGE 440.6}
(Tanagrae?)
\begin{greek}
Μνασάλκε̄ς ποίε̄σε.
\end{greek}
\end{versi}

\begin{versi}{DGE 440.7}
(Tanagrae)
\begin{greek}
Γαμε̄́δε̄ς ἐπόε̄σε.
\end{greek}
\end{versi}

\begin{versi}{DGE 440.10}
(Tanagrae)
\begin{greek}
Δε̄μοθέ(ρ)ρε̄ς hιαρὸν Ἀπό(λ)λο̄νος Καρυκε̄ϝίο̄.
\end{greek}
\end{versi}

\begin{versi}{DGE 440.13.A}
(Tanagrae)
\begin{greek}
Ϙυ(λ)λοατία ε̄̓μί .
\end{greek}
\end{versi}

\begin{versi}{DGE 440.13.B}
\begin{greek}
Φε(τ)τάλα καλά.
\end{greek}
\end{versi}
\textit{1-4. 13 de adi. possess. cf. Somsen BphW 1904,999sq. RhM 59,485. Wackernagel Mélanges Saussure 1908, 137sqq. Fraenkel IF 28, 229sq. BeD 1, 109.297. 6 \textgreek{π}.: argumentum omissum in prosa oratione inauditum; nam aliter se habet \textgreek{κάθθε̄κε} nr. 647a; quamquam est \textgreek{ἔον} in nr. 644 (cf. BeD 1, 80). 7 \textgreek{Γαμε̄́δε̄ς}: cf. \textgreek{Γάδωρος}, \textgreek{-τιμος} Sadée 98sq.; Sittig 80sq. ubi alii. 9 cf. Solmsen RhM 53, 137sq. 10 \textgreek{Δε̄μο}.: \textgreek{ε̄ } alienam originem prodit. 11 Καρ. cognomen Mercurii P. \textgreek{ἀπ.}: cf. Kretschmer ÖJ 3, 137 et lesb.  \textgreek{τοῖς ἀπάρχαισι} IG VII 2, 68, 5 thas. \textgreek{ἀπαρχή} ad nr. 780; BeAe p.14; v. Wilamowitz Griech. Trag. II (1900), 205; \textgreek{π} certa, quamvis \textgreek{γ} legi possit.}

\hrulefill

\subsubsection{DGE 444}
\textit{Lapis niger inventus Aulide; nunc Tanagrae. V. — Haussoullier BH 2, 590. IGA 234. Rob 225. DI 907. IG VII 547. Mi 764.}
\begin{versi}{DGE 444}
\begin{greek}
Μυλλιχιδάο̄ν.
\end{greek}
\end{versi}
\textit{``Consentaneum est, Myllichidas gentem fuisse, cuius fundo, quem ad sacellum aliquod gentilicium pertinuisse conieceris, hic terminus impositus erat'' IG. Cf. tit. Lebadeensem \textgreek{ἱαρὸν | Ἀρτάμι|δος Ποδι|αδάων} (Vollgraff BH 25, 365) et.}

\hrulefill

\subsubsection{DGE 449}
\textit{In Amphiaraio Oropiorum. Tabula marmoris candidi. III p. post. — Proxenia. — \textgreek{Λεόναρδος} EA 1891, 97/100. 1919, 78sq. IG VII 4259. Mi 220.}
\begin{versi}{DGE 449}
\begin{greek}
Χαροπίνω ἄρχοντος, Εὐφρ{\verso[1]}αίνων Πτωιοδώρου Ὀρώπι{\verso}ος ἔλεξε· δεδόχθη τοῖ δάμ{\verso}οι, Ἀπολλόδωρον Φρουνίχου Ἀθανῆον πρόξενον εἶμ{\verso}εν κὴ (ο)ἰκίας κὴ ἀσφάλιαν κὴ ἀσουλίαν κὴ πολέμω {\verso} ἰόντος κὴ ἰράνας κὴ κατὰ {\verso} γᾶν κὴ κατὰ θάλατταν κὴ τ{\verso}ἆλλα πάντα ὅσα \ladd{κ}{\K}ὴ τοῖς λο{\verso}{\K}ιποῖς προξένοις κὴ εὐεργέτηις τῶ κοινῶ Βοιωτῶν.
\end{greek}
\end{versi}
\textit{Cf. similia decreta ab Oropiis ad \textgreek{κοινὸν Βοιωτῶν} rogata IG VII 280. 4260 (Boeotica dialecto scripta esse decreta a Tanagraeis Opuntio Plataeensi rogata et Oropi insculpta IG VII 283. 290. 393. 4261 mirum non est); decreta oppidi Oropiorum dialecto Eretriaca (cf. nr. 811) scripta sung, postea lingua communi (cf. Buttenwieser IF 28, 83sq.). — vs. 9 (\textgreek{ο}): κ 1. 15 \textgreek{-ηις Λεον}. 1919.}

\hrulefill

\subsubsection{DGE 450}
\textit{Aegosthenis. In eodem lapide ac nr. 161. — honores. — Le Bas II 1. DI 1145. IG VII 207. Mi 170. Cf. Buck ClassPH 8,147.}
\begin{versi}{DGE 450}
\begin{greek}
Νικίας Διονυσίου ἔλεξε· προβεβωλευμένον {\verso[1]} \ladd{ε}{\K}ἶμεν αὐτοῖ· ἐπιδή ἐστι τῆ πόλι Σιφείων προ{\verso}\ladd{υ}{\K}πάρχωσα εὔνοια ἐκ προγόνων κὴ ἐν προεδρίαν {\verso} καλῖ ἁ πόλις Ἠγοσθενιτάων ὁπόττοι κα παρίωνθι Σιφείων, κατὰ ταὐτὰ δὲ κὴ τοὶ Σιφεῖες τὰς {\verso} αὐτὰς τιμὰς ἐκτεθήκανθι Ἠγοσθενίτης κὴ ἐ{\verso}\ladd{π}{\K}ὶ τὰς κοινὰς συνόδως καλέονθι τὼς παργινυ{\verso}μένως· ὅπωτ ὦν πανερὸν ἴει, ὅτι τὰν ὁμόνοι{\verso}αν διαφυλάττι τὰν ἐκ τῶν προγόνων παρδοθεῖσαν ἁ πόλις Ἠγοσθενιτάων πὸτ τὰν πόλιν Σιφε{\verso}ίων, δεδόχθη τοῖ δάμοι· ὁπόττοι κα παργινύ{\verso}ωνθη Σιφείων ἐν τὰς κοινὰς θυσίας, ἃς δαΐζοι ἁ πό{\verso}\ladd{λ}{\K}ις, ὑπαρχέμεν αὐτοῖς καθάπερ κὴ τοῖς πολ{\verso}ίτης· τοὶ δὲ πολέμαρχοι, ἐπί κα τὸ ψάφισμα κου\ladd{ρ}{\K}ωθείει, ἀνγραψάνθω τὸ ψάφισμα ἐν στάλαν ἐ\ladd{ν} {\verso} τοῖ Μελαμποδείοι.
\end{greek}
\end{versi}
\textit{Cf. ad nr. 181. Vs. 8 \textgreek{οπωτ} (quot solus vidit Le Bas) mero errore lapicidae an falsa quam vocant analogia (\textgreek{πρὸς}: \textgreek{πὸτ} = \textgreek{ὅπως}: \textgreek{ὅπωτ}).}

\section{Tituli Tanagraei}
\subsubsection{DGE 451}
\textit{Tabula lapidis nigri. — Tit. sepulcralis Tanagraeorum in bello (probabiliter in pugna Tanagraea a. 426) occisorum — \textgreek{Κουμανούδης Ἀθήν.} 4, 213. Im 85. DI 914. IG VII 585. Mi 615. Hicks 28. So 14. Cf. Kirchh. A. 141}
\begin{versi}{DGE 451.A}
\begin{greek}
-ος {\verso[1]} -ος {\verso} -δας {\verso} -χος Ἀρ\ladd{ιστο}{\K}τέλε̄ς (?) {\verso} Μοέ(ρ)ιχο(ς) {\verso} Ἀριστόθοενος {\verso} Διόπομπος {\verso} Δαλιάδας Ἀβαεόδο̄ρος {\verso} Λάκο̄ν {\verso} Παθσανίας {\verso} Πίθαρχος {\verso} Δαμότιμος Νικίας. {\verso} Φανόδαμος \; Ἐρετριεύς. {\verso} \ladd{Μ}ύννος \; Ἐρετριεύς.
\end{greek}
\end{versi}

\begin{versi}{DGE 451.B}
\begin{greek}
Γοθθίδας {\verso[1]} Μισσ(θ)ίδας {\verso} σαμίας {\verso} Πυθάνγελος Ἀρισστόδαμος {\verso} ΧαρΟνδας {\verso} Εὐαγοντίδας {\verso} Λακριδίο̄ν {\verso} Δαμομέλο̄ν Διάκριτος {\verso} Μελίτο̄ν {\verso} Μορυχίδας {\verso} Βαχχυλίδας {\verso} Ἀριόμναστος Μεγαλῖνος.
\end{greek}
\end{versi}
\textit{6 \textgreek{Χαρόνδας} aut \textgreek{Χαρώνδας}; cf. Sadée 52. 105.}

\begin{versi}{DGE 451.C}
\begin{greek}
Χαβᾶς {\verso[1]} Αἰσχίνας {\verso} Πυρραῖος {\verso} Ἀχύλλε̄ Ϝεργαέντος {\verso} Φάλ(α)ρις {\verso} Ἐράτο̄ν {\verso} Ἀμινοκλέε̄ς {\verso} Μάτρο̄ν Ὀνατορίδας {\verso} Φιλοχάρε̄ς {\verso} Ἀπολλόδο̄ρος {\verso} Μεγγίδας {\verso} Ἱσστιαΐδας Θεόζοτος.
\end{greek}
\end{versi}

\begin{versi}{DGE 451.D}
\begin{greek}
Κοέρανος {\verso[1]} Ἀφρόδιτος {\verso} Σαγυθινίδα\ladd{ς} {\verso} Σαυγένε̄ς Εὐκλίδας {\verso} Δαμόξενος {\verso} ΧαρΟνδας {\verso} Καφισοφάο̄ν {\verso} Καλλικράτε̄ς Ϝισοκλέε̄ς {\verso} Χοερίλος {\verso} Σάρβαλος {\verso} Γόργος {\verso} Ἀπολλόδο̄ρος Βυλιά\ladd{δας} {\verso} Ἀμεύσ\ladd{ιππ}{\K}ος.
\end{greek}
\end{versi}
\textit{3 \textgreek{Σαγυθ}: cf. Solmsen IF 30, 42.}

\hrulefill

\subsubsection{DGE 452}
\textit{Titt. sepulcrales privati litteratura vetustiore scripti. — IG VII 589 sqq.}

\begin{versi}{DGE 452.1}
\begin{greek}
ἐπὶ Ἀθανοδο̄́ρα\ladd{ε}{\K}.
\end{greek}
\end{versi}
\textit{IG 589 (de \textgreek{ἐπί} c. dat. cf. ad nr. 348; IG VII p. 182.).}

\begin{versi}{DGE 452.2}
\begin{greek}
ἐπ\ladd{ὶ Ἀ}{\K}μεινοκλείαε.
\end{greek}
\end{versi}
\textit{IG VII 590.}

\begin{versi}{DGE 452.3}
\begin{greek}
\ladd{ἐπὶ} Ἀνφάλκει.
\end{greek}
\end{versi}
\textit{IG VII 591.}

\begin{versi}{DGE 452.4}
\begin{greek}
ἐπὶ Εὐχσενίδα\ladd{ε}{\K}.
\end{greek}
\end{versi}
\textit{IG VII 592.}

\begin{versi}{DGE 452.5}
\begin{greek}
ἐπὶ Ϝhεχαδάμοε ε̄̓μί.
\end{greek}
\end{versi}
\textit{IG VII 593. Im 83.}

\begin{versi}{DGE 452.6}
\begin{greek}
ἐπὶ Δυσανίαε Hιαρίδα\ladd{ο}{\K}.
\end{greek}
\end{versi}
\textit{IG VII 596.}

\begin{versi}{DGE 452.7}
\begin{greek}
ἐπὶ Πολυαράτοε ε̄̓μί.
\end{greek}
\end{versi}
\textit{IG VII 599.}

\begin{versi}{DGE 452.8}
\begin{greek}
ἐπὶ Πραύ{\verso[1]}χαε.
\end{greek}
\end{versi}
\textit{IG VII 600. Im 82.}

\begin{versi}{DGE 452.8\super{a}}
\begin{greek}
ἒπ Πυλαρέτοι.
\end{greek}
\end{versi}
\textit{IG VII 601. — Cf. Sadée 96.}

\begin{versi}{DGE 452.9}
\begin{greek}
ἐπὶ Πυλιμιάδαε.
\end{greek}
\end{versi}
\textit{IG VII 602. — Cf. WSchulze KZ 33, 243; Sadée 9sq.}

\begin{versi}{DGE 452.10}
\begin{greek}
ἐπὶ Φαε̄νίδι ε̄̓μί Hε …\footnote{h\textgreek{ε...} in IG VII 605.}
\end{greek}
\end{versi}
\textit{IG VII 605.}

\begin{versi}{DGE 452.11}
\begin{greek}
ἐπὶ Ὀ̄κίβαε.
\end{greek}
\end{versi}
\textit{IG VII 606.}

\begin{versi}{DGE 452.12}
\begin{greek}
Ἀβαεόδο̄ρος.
\end{greek}
\end{versi}
\textit{IG VII 612.}

\begin{versi}{DGE 452.13}
\begin{greek}
Ἀθανογιτίς.
\end{greek}
\end{versi}
\textit{IG VII 613.}

\begin{versi}{DGE 452.14}
\begin{greek}
Ἀρνε̄σίχα.
\end{greek}
\end{versi}
\textit{IG VII 617.}

\begin{versi}{DGE 452.15}
(in cantharo)
\begin{greek}
Ἀσο̄́.
\end{greek}
\end{versi}
\textit{IG VII 618.}

\begin{versi}{DGE 452.16}
(in cyatho nigro)
\begin{greek}
Βελφίς.
\end{greek}
\end{versi}
\textit{IG VII 619.}

\begin{versi}{DGE 452.17}
\begin{greek}
Βο̄ϙᾶς.
\end{greek}
\end{versi}
\textit{IG VII 620.}

\begin{versi}{DGE 452.18}
(in cantharo nigro)
\begin{greek}
Δαλιόδο̄ρος
\end{greek}
(sin).
\end{versi}
\textit{IG VII 621.}

\begin{versi}{DGE 452.19}
\begin{greek}
Δεχσαρέτα.
\end{greek}
\end{versi}
\textit{IG VII 623.}

\begin{versi}{DGE 452.20}
\begin{greek}
Εὐγιτονίδα.
\end{greek}
\end{versi}
\textit{IG VII 3508 (nom.; cf. ad nr. 143).}

\begin{versi}{DGE 452.21}
\begin{greek}
Εὐϝαεν\ladd{έ}{\K}τα.
\end{greek}
\end{versi}
\textit{IG VII 3,510.}

\begin{versi}{DGE 452.22}
\begin{greek}
Θε(ρ)ιππίο̄ν.
\end{greek}
\end{versi}
\textit{630. Im 82.}

\begin{versi}{DGE 452.23}
\begin{greek}
Θιομνάστα.
\end{greek}
\end{versi}
\textit{IG VII 631.}

\begin{versi}{DGE 452.24}
\begin{greek}
Θραικία.
\end{greek}
\end{versi}
\textit{IG VII 633.}

\begin{versi}{DGE 452.25}
\begin{greek}
Hι(π)πάρχα.
\end{greek}
\end{versi}
\textit{IG VII 635.}

\begin{versi}{DGE 425.26}
\begin{greek}
Λάοτος.
\end{greek}
\end{versi}
\textit{IG VII 640.}

\begin{versi}{DGE 452.27}
\begin{greek}
Μελάντιχος.
\end{greek}
\end{versi}
\textit{IG VII 643.}

\begin{versi}{DGE 452.28}
\begin{greek}
Μισθίδ\ladd{ας}{\K}.
\end{greek}
\end{versi}
\text{IG VII 646}

\begin{versi}{DGE 452.29}
\begin{greek}
Χσενόκλια.
\end{greek}
\end{versi}
\text{IG VII 650}

\begin{versi}{DGE 452.30}
\begin{greek}
Πεδανγελίς.
\end{greek}
\end{versi}
\text{IG VII 3512}

\begin{versi}{DGE 452.31}
\begin{greek}
Πε̄λεξενίς.
\end{greek}
\end{versi}
\text{IG VII 654}

\begin{versi}{DGE 452.32}
\begin{greek}
Πισιδίκα.
\end{greek}
\end{versi}
\text{IG VII 655}

\begin{versi}{DGE 452.33}
\begin{greek}
Πρίϙο̄ν.
\end{greek}
\end{versi}
\text{IG VII 657}

\begin{versi}{DGE 452.34}
\begin{greek}
Πυρρῖνος.
\end{greek}
\end{versi}
\text{IG VII 659}

\begin{versi}{DGE 452.35}
\begin{greek}
Φε(τ)ταλός.
\end{greek}
\end{versi}
\text{IG VII 664}

\begin{versi}{DGE 452.36}
\begin{greek}
Φίθο̄ν.
\end{greek}
(sin.)
\end{versi}
\text{IG VII 665}

\hrulefill

\subsubsection{DGE 453}
\text{Marmor longum nigrum. V. — Dedicatio. — Leake NGr nr. 71. CIG 1599. IGA 153. DI 869. IG VII 550}
\begin{versi}{DGE 453}
\begin{greek}
ΑἐσχρΟνδας Αἐγίτ\ladd{αο}{\K} {\verso[1]} Διονύσοε.
\end{greek}
\end{versi}
\textit{\textgreek{Αἐγίτ{\lbrk}αο} IGA; cf. \textit{Αἰγίτας} Theb. BH 22, 270 (Sadée 71, 1).}

\hrulefill

\subsubsection{DGE 454}
\textit{Marmor caeruleum. V. — Terminus. — Robert AZ 33, 160. IGA 170. DI 883. IG VII 546. Mi 765.}
\begin{versi}{DGE 454}
\begin{greek}
Ἀρτάμι{\verso}δος
\end{greek}
\end{versi}

\hrulefill

\subsubsection{DGE 455}
\textit{Lapis sepulcralis pornius cum anaglypho duorum virorum nundorum qui inter se amplectuntur. V in. — Κουμανούδης Ἀθην. 2, 404 sq. Körte AM 3, 309/11. IGA 265. Rob 219. E 484. DI 875. IG VII 579.}
\begin{versi}{DGE 455}
\begin{greek}
Ἀμφάκλε̄ς \ladd{ἔ}{\K}στασ’ ἐπὶ Κιτύλοι ε̄̓{\verso[1]}δ’ ἐπὶ Δέρμυι. {\verso} Κιτύλος. {\verso} Δέρμυς.
\end{greek}
\end{versi}

%\subsubsection{DGE 811}
%\textit{In Amphiario Oropium. Tabula marmoris candidi. Ante 338. — Proxenia. — \textgreek{Β Λεονάρδος} EA 1891, 107sq. IG VII 4250. Mi 202. DS 124 ³258. DI 5338.}
%\begin{versi}{DGE 811}
%\begin{greek}
%θεός.{\verso[1]}
%Δρίμων ἔλεξε· ἔδοξε {\verso}
%τεῖ ἐκκλησίει· ἀγαθεῖ τύχει, {\verso}
%Ἀμύνταν Ἀντιόχου Μακε{\verso}%
%δόνα πρόξενον εἶν Ὠρωπί{\verso}%
%ων καὶ εὐεργέτην· ἀτέλειαν {\verso}
%δὲ εἶν καὶ ἀσυλίαν καὶ πολέμου {\verso}
%καὶ εἰρήνης καὶ γῆς καὶ οἰκίης {\verso}
%ἔνκτησιν αὐτῶι καὶ ἐκγόνοις.  :
%\end{greek}
%\end{versi}
%\textit{vs. 2 \textgreek{ἔλεξε}: boeot. pro \textgreek{εἶπε}; Oropus a. 366/38 foederis erat Boeotici (IG). 3 \textgreek{ἐκκλ}.: deest rhotacismus sicut \textgreek{ἔνκτησιν} vs. 9. (IG). 4 Amyntas a. 333 mortuus est (IG). 5 \textgreek{εἶν}: cf. ad nr. 811, 32.}