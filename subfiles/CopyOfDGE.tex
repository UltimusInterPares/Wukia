\chapter{Selections from Schwyzer, Eduard: \textit{Dialectorum Graecarum Exempla Epigraphica
Potiora}}

\section{Tituli Megarici}
\subsection{DGE 155}
Megaris in muro Olympiei. Lapis. III in. — Proxenia. — Ross BerlBer 1844,
160 sq.\ DI 3005. IG VII 8.

\begin{versi}{DGE 155}
  \begin{greek}
    ἐπιδὴ Ἀγαθοκλῆς Ἀρχιδάμου {\verso[1]} Βοιώτιος εὔνους ἐὼν διατελεῖ
    {\verso} καὶ εὐεργέτας τοῦ δάμου τοῦ {\verso} Μεγαρέων, ἀγαθᾶι τύχαι
    δεδόχθαι τᾶι βουλᾶι καὶ τῶι δάμωι· {\verso} πρόξενον αὐτὸν εἶμεγ καὶ
    ἐκ{\verso}γόνους αὐτοῦ τᾶς πόλιος τᾶς {\verso} Μεγαρέωγ κὰτ τὸν νόμον·
    εἶμεν {\verso} δὲ αὐτῶι καὶ οἰκίας ἔμπασιν καὶ προεδρίαν ἐμ πᾶσι τοῖς
    ἀγῶ{\verso}σιν οἷς ἁ πόλις τίθητι· ἀγγραψά{\verso}τω δε̃ τ̃ο δόγμα τόδε
    ὁ γραμματεὺς τοῦ δάμου ἐν στάλαι λιθί{\verso}ναι καὶ ἀνθέτω εἰς τὸ
    Ὀλυμπιεῖον. βασιλεὺς Πασιάδας· ἐστρατά{\verso}γουν Διονύσιος Πυρρίδα,
    Δαμέ{\verso}ας Ματροκλέος, ἀντίφλιλος Σμά{\verso}χου, Μνασίθεος
    Πασίωνος, Ἑρκίων {\verso} Τέλητος· γραμματεὺς βουλᾶς καὶ δάμου Ἵππων
    Παγχάρεος.
  \end{greek}
\end{versi}

%\section{Acta Foederis Boeotici et Tituli Incertae Originis}

\section{Tituli Tanagraei}
\subsection{DGE 459}
Basis quadrata marmoris caerulei. III ex. — Proxeniae. — Rober H 11,
97/103. DI 936/8. IG VIII 505/7. Mi 223 (3). DittOr 80 (3)

\begin{versi}{DGE 459.1}
  \begin{greek}
    Εἱρίαο ἄρχοντος μεινὸς Δαματρίω νιομεινίη, ἐπεψάφιδδε {\verso[1]}
    Γυνόππαστος Ἀμινίωνος, Ἐπιχαρίδας Φύλλιος ἔλεξε, δε{\verso}δόχθη τοῖ
    δάμοι· πρόξενον εἶμεν κὴ εὐεργέταν τᾶς πόλιος {\verso} Ταναγρήων πέλοπα
    Δεξίαο Νιαπολίταν αὐτὸν κὴ ἐσγόνως, {\verso} κὴ εἶμεν αὐτοῖς γᾶς κὴ
    ϝυκίας ἔππασιν κὴ ἀσφάλιαν κὴ ϝισοτέλιαν \ladd{κὴ}{\K} {\verso}
    ἀσουλίαν κὴ πολέμω κὴ ἰράνας ἰώσας κὴ κατὰ γᾶν κὴ κατὰ θάλατταν
    {\verso} \ladd{κ}{\K}ὴ τἆλλα πάντα καθάπερ τοῖς ἄλλοις προξένοις κὴ
    εὐεργέτης.
  \end{greek}
\end{versi}

\begin{versi}{DGE 459.2}
  \begin{greek}
    Ξεναρίστω ἄρχοντος μεινὸς Ἀλαλκομενίω πετράδι ἀπιόντος, {\verso[1]}
    ἐπεψάφιδδε Ἀχηός, Ἀπολλόδωρος Καφισίαο ἔλεξε, δεδόχθη {\verso} τοῖ
    δάμοι· πρόξενον εἶμεν κὴ εὐεργέταν τᾶς πόλιος Ταναγρήων {\verso}
    Ἀντίγονον Ἀσκλαπιάδαο Μακεδόνα αὐτὸν κὴ ἐσγόνως, κὴ εἶμ\ladd{εν}{\K}
    {\verso} αὐτοῖ γᾶς κὴ ϝοικίας ἔππασιν κὴ ἀσφάλιαν κὴ ἀσουλίαν κὴ πολέμω
    {\verso} κὴ ἰράνας ἰώσας κὴ κατὰ γᾶν κὴ κατὰ θάλατταν κὴ τἆλλα πάντα
    {\verso} καθάπερ τοῖς ἄλλοις προξένοις κὴ εὐεργέτης.
  \end{greek}
\end{versi}

\begin{versi}{DGE 459.3}
  \begin{greek}
    Εὐξιθίω ἄρχοντος μεινὸς Δαματρίω ὀγδόη ἱσταμένω, ἐπεψάφιδδε καφισίας,
    {\verso[1]} Μειλίων Ἀφροδίτω ἔλεξε, δεδόχθη τὶ δάμοι· πρόξενον εἶμεν κὴ
    εὐεργέταν {\verso} τᾶς πόλιος Ταναγρήων Σωσίβιον Διοσκουρίδαο
    Ἀλεξανδρεῖα αὐτὸν κὴ ἐσγόν\ladd{ως}{\K} {\verso} κὴ εἶμεν αὐτοῖς γᾶς κὴ
    οἰκίας ἔππασιν κὴ ϝισοτέλιαν κὴ ἀσφάλιαν κὴ ἀσουλία\ladd{ν}{\K}
    {\verso} κὴ πολέμω κὴ ἰράνας ἰώσας κὴ κατὰ γᾶν κὴ κατὰ θάλατταν
    \ladd{κὴ}{\K} τἆλλα πά\ladd{ντα}{\K} {\verso} καθάπερ τοῖς ἄλλοις
    προξένοις κὴ εὐεργέτης.
  \end{greek}
\end{versi}

Sosibius Ptolemaei IV Philopatoris (222/205) familiaris; cf.\ nr.\ 525.

%\hrulefill

\subsection{DGE 460}
In eodem lapide quo nr.\ 457. III ex. — Proxeniae. —
\textgreek{Κουμανούδης Ἀθήν}. 4, 210sq. Di 951/2. IG VII 517/8. Mi 224 (2).

\begin{versi}{DGE 460.1}
  \begin{greek}
    Ἀριστοκλίδαο ἄρχοντος μεινὸς Θουίω νευμεινίη, {\verso[1]} κατὰ δὲ τὸν
    θιὸν Ὁμολωΐω ἑσκηδεκάτη, ἐπεψάφιδδε Ἀγάθαρ{\verso}χος, Εὔνοστος
    Μελίτωνος ἔλεξε· δεδόχθη τῦ δάμυ· πρό{\verso}ξενον εἶμεν κὴ εὐεργέταν
    τᾶς πόλιος Ταναγρήων Διω{\verso}νούσιον Θιοφίδιος Δαματρεῖα αὐτὸν κὴ
    ἐσγόνως κὴ εἶμεν {\verso} αὐτῦς γᾶς κὴ ϝυκίας ἔππασιν κὴ ϝισοτέλιαν κὴ
    {\verso} ἀσφάλιαν κὴ ἀσουλίαν κὴ πολέμω κὴ ἰράνας ἰώ{\verso}σας κὴ
    κατὰ γᾶν κὴ κατὰ θάλατταν, κὴ τὰ ἄλλα {\verso} πάντα καθάπερ τῦς ἄλλυς
    προξένυς κὴ εὐεργέτης.
  \end{greek}
\end{versi}

\begin{versi}{DGE 460.2}
  \begin{greek}
    Νικίαο ἄρχοντος μεινὸς Ἀλαλκομένου ἕκ\ladd{τη}{\K} ἀπιόντος,
    {\verso[1]} ἐπεξάφιδδε Εὐκτείμων, Θιόπομπος Εὐνόμω ἔλεξε,
    δε{\verso}δόχθη τῦ δάμυ· προξένως εἶμεν κὴ εὐεργέτας τᾶς
    πόλιος{\verso}<ιος> Ταναγρήων Φιλοκράτην Ζωΐλω. Θηραμένην Δαματρίω,
    Ἀπολλοφάνην Ἀθανοδότω Ἀντιοχεῖας τῶν πὸδ Δάφνῃ, αὐτὼς {\verso} κὴ
    ἐσγόνως, κὴ εἶμεν αὐτῦς γᾶς κὴ ϝυκίας ἔππασιν κὴ {\verso} ϝισοτέλιαν κὴ
    ἀσφάλιαν κὴ ἀσουλίαν κὴ πολέμω {\verso} κὴ ἰράνας ἰώσας κὴ κατὰ γᾶν κὴ
    κατὰ θάλατταν, κὴ τὰ {\verso} ἄλλα πάντα καθάπερ τῦς ἄλλυς προξένυς κὴ
    εὐεργέτης.
  \end{greek}
\end{versi}

\section{Tituli Eretriae et Oropi, Styrorum, Orei}

\subsection{DGE 811}
In Amphiaraio Oropiroum. Tabula marmoris albi. 387/77 (cf.\ DS)? paulo ante
338? Wilh. — Lex sacra. — \textgreek{Β Λεονάρδος} EA 1885, 93/8. 1917,
231/6 (accuratissime delineavit et explicavit.) IG VII 235. Ho 3, 25.
Mi 698. DS 589 \textsuperscript{3}1004. DI 5339. LS 2, 65. So 57. Bu 14.
Cf.\ v. Wilamowitz H 21, 91/115; B Keil H 25, 599/606
(de dial. Orop.; cf.\ nr.\ 449).

\begin{versi}{DGE 811}
  \begin{greek}
    θεοί. {\verso[1]} τὸν ἱερέα τοῦ Ἀμφιαράου φοιτᾶν εἰς τὸ ἱερό{\verso}ν,
    ἐπειδὰν χειμὼν παρέλθει, μέχρι ἀρότου ὥρ{\verso}ης, μὴ πλέον
    διαλείποντα ἢ τρεῖς ἡμέρας καὶ μένειν ἐν τοῖ ἱεροῖ μὴ ἔλαττον ἢ δέκα
    ἡμέρα{\verso}ς τοῦ μηνὸς ἑκ(ά)στο̄. \: καὶ ἐπαναγκάζειν τὸ
    ν{\verso}εωκόρον τοῦ τε ἱεροῦ ἐπιμελεῖσθαι κατὰ τὸ{\verso}ν νόμον καὶ
    τῶν ἀφικνε̄μένων εἰς τὸ ἱερόν. {\verso} ἂν δέ τις ἀδικεῖ ἐν τοῖ ἱεροῖ 
    ξένος ἢ δημότης, ζημιούτω ὁ ἱερεὺς μέχρι πέντε δραχμέων {\verso} χυρίως
    καὶ ἐνέχυρα λαμβανέτω τοῦ ἐζημιωμ{\verso}ένου· ἂν δ'' ἐκτίνει τὸ
    ἀργύριον, παρεόντος το̃ {\verso} ἱερέος ἐμβαλέτω εἰς τὸν θησαυρόν. \;
    δικάζει{\verso}ν δὲ τὸν ἱερέα, ἄν τις ἰδίει ἀδικηθεῖ ἢ τῶν ξένων ἢ τῶν
    δημοτέων ἐν τοῖ ἱεροῖ, μέχρι τριῶν {\verso} δραχμέων, τὰ δὲ μέζονα,
    ἡχοῖ ἑκάστοις αἱ δίκ{\verso}αι ἐν τοῖς νόμοις εἰρῆται, ἐντο̃θα γινέσθων
    {\verso} προσκαλεῖσθαι δὲ καὶ αὐθημερὸν περὶ τῶν ἐ{\verso}ν τοῖ ἱεροῖ
    ἀιδίων· ἂν δὲ ὁ ἀντίδικος μὴ συνχωρεῖ, εἰς τὴν ὑστέρην ἡ δίκη τελείσθω.
    \; ἐπαρ{\verso}χὴν δὲ διδοῦν τὸμ μέλλοντα θεραπεύεσθαι ὑ{\verso}πὸ τοῦ
    θεοῦ μὴ ἔλαττον ἐννέ'' ὀβολοὺς δοκίμου ἀργ{\verso}υρίου καὶ ἐμβάλλειν
    εἰς τὸν θησαυρὸν παρε{\verso}όντος τοῦ νεωκόρου. κατεύχεσθαι δὲ τῶν
    ἱερῶν καὶ ἐπ{\verso}ὶ τὸν βωμὸν ἐπιτιθεῖν, ὅταν παρεῖ, τὸν ἱερέα,
    {\verso} ὅταν δὲ μὴ παρεῖ, τὸν θύοντα, καὶ τεῖ θυσίει α{\verso}ὐτὸν
    ἑαυτοῖ κατεύχεσθαι ἕκαστον, τῶν δὲ δη{\verso}μορίων τὸν ἱερέα. τῶν δὲ
    θυομένων ἐν τοῖ ἱεροῖ πάντων τὸ δέρμα. θύειν δὲ ἐξ{\verso}εῖν ἅπαν ὅτι
    ἂν βόληται ἕκαστος· τῶν δὲ κρεῶ{\verso}ν μὴ εἶναι ἐκφορὴν ἔξω τοῦ
    τεμένεος. τοῖ δὲ {\verso} ἱερεῖ διδοῦν το̄̀ς θύοντας ἀπὸ τοῦ ἱερή
    υ ἑκ{\verso}άστο̄ τὸν ὦμον, πλὴν ὅταν ἡ ἑορτὴ εἶ· τότε δὲ ἀπὸ τῶ
    δημορίων λαμβανέτω ὦμον ἀφ'' ἑκάστου {\verso} τοῦ ἱερήου. ἐγκαθεύδειν δὲ
    τὸν δειόμενο{\verso}ν{\verso} — {\verso} πειθόμ{\verso}ενον τοῖς
    νόμοις. τὸ ὄνομα τοῦ ἐγκαθεύδοντος, ὅταν ἐμβάλλει τὸ ἀργύριον,
    γράφεσθαι τ{\verso}ὸν νεωκόρον καὶ αὐτοῦ καὶ τῆς πόλεος καὶ
    ἐκ{\verso}τιθεῖν ἐν τοῖ ἱεροῖ γράφοντα ἐν πετεύροι σ{\verso}κοπεῖν τοῖ
    βολομένοι. ἐν δὲ τοῖ κοιμητηρίο{\verso}ι καθεύδειν χωρὶς μὲν το̄̀ς
    ἄνδρας, χωρὶς δὲ τὰς γυναῖκας, τοὺς μὲν ἄνδρας ἐν τοπι πρὸ
    ἠ{\verso}\ladd{ο̃}ς τοῦ β\ladd{ω}{\K}μοῦ, τὰς \ladd{δ}{\K}ὲ γυναῖκας ἐ
    τοῖ πρὸ h(ε)σπέ{\verso}\ladd{ρης – – τὸ κοιμ}ητήριον τοὺς
    ἐν{\verso}καθεύδοντας – – τὸν δὲ θεὸν – – –.
  \end{greek}
\end{versi}

Omissa rasurae (vss.\ 6. 22. 24sq.\ 30. 37sq.), spatia (nihil ad rem
pertinentia), reliquiae vss. 49/56. vs.\ 6 (\textgreek{α}): \textgreek{σ}
in 1.\ 8 \textgreek{ἀφικνε̄μ}. Boeotorum more.\ DS, \textgreek{ἀφικνε(ο)μ}.
priores. 13 \textgreek{ἐμβαλέτω} defendit Wilhelm ÖJ 14, 248,
-(\textgreek{λ})\textgreek{λ}- priores; cf.\ \textgreek{ἐμβάλωντι} nr.\ 74,
87; \textgreek{ἐπιβαλὸν} nr.\ 353 A 27sq.\ et alia apud Leon.\ 1917.\ 16
\textgreek{ἑκάστοις}: generis masc.; cf.\ Wilhelm WienBer 179, 6, 6/8.\ 17
\textgreek{ἐντο̃θα}: cf.\ \textgreek{κατὰ} \textgreek{τοῦτα}
nr.\ 808, 42.\ 19 \textgreek{ἀδικίων} So.\ DS\textsuperscript{3}, -\textgreek{ιῶν} priores.
22 \textgreek{ἐννέ''} \textgreek{ὀβολοὺς} Leon. 1917,
\textgreek{ἐννεοβόλου} priores; \textgreek{ἐνν}. \textgreek{ο}.
\textgreek{δοκί}- in rasura. 25
\textgreek{κατεύχ}.: \textgreek{κατάρχεσθαι} coniecit Stengel H 43, 464. 31
\textgreek{βόληται}: cf.\ nr.\ 808, 31; \textgreek{βολόμενον} est in vs.\ 
56. 32 \textgreek{εἶναι} sed \textgreek{ἐξεῖν} 30sq., \textgreek{εἶν} in
tit.\ Olynthio DI 5285 (e.\ gr.\ A 3sq.\ \textgreek{συμμάχους εἶν ἀλλήλοισι
κατὰ πάντας ἀνθρώπου\ladd{ς}{\K} | ἔτεα πεντήκοντα}, B 7sq.\ 
\textgreek{καὶ τῶν ἄλλων ἐξαγωγὴν δὲ εἶν καὶ δι}
<\textgreek{α}>|\textgreek{αγωγήν}; 389/83\textsuperscript{a}) in
tit.\ Chio Delphis reperto DS\textsuperscript{3} 402, 37 \textgreek{τὴν δὲ
προγ{\rbrk}ραφὴν εἶν εἰς φυλακήν} (a. 276; cf.\ Wilhelm WienAnz 1922,
7sq.). 46 H (i.e.\ h\textgreek{ε} sec.\ Wil.) l. — Quominus tit.\ V saeculo
addicatur, litteratura impedimento est. Wilh.

%\hrulefill

\subsection{DGE 812}
In Amphiario Oropium. Tabula marmoris candidi. Ante 338. — Proxenia. —
\textgreek{Β Λεονάρδος} EA 1891, 107sq. IG VII 4250. Mi 202. DS 124
\textsuperscript{3}258.
DI 5338.

\begin{versi}{DGE 812}
  \begin{greek}
    θεός. {\verso[1]} Δρίμων ἔλεξε· ἔδοξε {\verso} τεῖ ἐκκλησίει· ἀγαθεῖ
    τύχει, {\verso} Ἀμύνταν Ἀντιόχου Μακε{\verso}δόνα πρόξενον εἶν
    Ὠρωπί{\verso}ων καὶ εὐεργέτην· ἀτέλειαν {\verso} δὲ εἶν καὶ ἀσυλίαν καὶ
    πολέμου {\verso} καὶ εἰρήνης καὶ γῆς καὶ οἰκίης {\verso} ἔνκτησιν αὐτῶι
    καὶ ἐκγόνοις.
  \end{greek}
\end{versi}

vs.\ 2 \textgreek{ἔλεξε}: boeot.\ pro \textgreek{εἶπε}; Oropus a.\ 366/38
foederis erat Boeotici (IG). 3 \textgreek{ἐκκλ}.: deest rhotacismus sicut
\textgreek{ἔνκτησιν} vs.\ 9.\ (IG).\ 4 Amyntas a.\ 333 mortuus est (IG).\ 5
\textgreek{εἶν}: cf.\ ad nr.\ 811, 32.

\begin{versi}{DGE 812\textsuperscript{a}}
  \begin{greek}
    θεοί. {\verso[1]} Σωφίλου ἱερέος, {\verso} Ἀντιφάνης ἔλεξεν. {\verso}
    δεδόχθαι τεῖ ἐκκλησίε\ladd{ι}{\K} Μικυθίωνα Σφαγγειλαῖο\ladd{ν}
    {\verso} εὐεργέτην ἀναγράψαι {\verso} Ὠρωπίων αὐτὸν καὶ
    ἐκγό{\verso}νους, εἶναι δὲ καὶ ἀσυλίη\ladd{ν}{\K} {\verso} καὶ ἀτέλειαν
    καὶ πολέμου ἐόντος καὶ εἰρήνης {\verso} καὶ αὐτοῖ καὶ ἐκγόνοις.
  \end{greek}
\end{versi}

vs.\ 5 \textgreek{Σφ}. i. e. \textgreek{Συαγγελέα} (\textgreek{Συάγγελα}
Cariae oppidum, postea \textgreek{Θεάγγελα} vocatum).